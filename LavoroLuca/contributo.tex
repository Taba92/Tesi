\documentclass[Tesi.tex]{subfiles}

\begin{document}

\section{ESTENSIONE DI CAUDER CON ERROR-HANDLING BASATO SUI LINK}
In questa sezione darò una panoramica dell'estensione effettuata su Cauder, partendo da quali espressione vengono aggiunte al linguaggio, quali strutture dati vengono aggiunte e come le nuove espressioni modifichino tali strutture dati.\\
La comunicazione di Erlang viene eseguita tramite \textit{segnalazione asincrona}. Tutte le diverse entità in esecuzione, come processi e porte, comunicano tramite segnali \textit{asincroni}\cite{erlangCommunication}.
Il segnale più comunemente usato è il \textit{messaggio}. Altri segnali comuni sono i segnali di uscita, collegamento, scollegamento, monitoraggio e demonitoraggio.\\
In questa tesi useremo il termine segnale per riferirci unicamente ai segnali di uscita.
In BEAM, un \textit{link} \cite{erlangLinks} tra due processi può essere visto come un canale bidirezionale in cui vengono trasmessi gli errori che avvengono nei rispettivi processi linkati.
Un link tra due processi può essere creato tramite apposite funzioni.\\
\footnote{Può esistere un solo link tra due processi, quindi qualsiasi tentativo di crearne ulteriori, non produrrà nessun effetto.}
In questa tesi verrà trattata solo la funzione \textit{spawn\_link}, che sarà spiegata più in dettaglio nella sezione successiva.\\
Quando un processo si arresta, sia in modo anomalo (esempio: errore di un pattern match) che in modo normale (il codice termina), i segnali di uscita vengono inviati a \textit{tutti} i processi a cui è attualmente linkato il processo morente.
A seconda di come sia impostato il \textit{flag trap\_exit} di un processo ricevente, il segnale viene gestito in modo differente. Questo flag viene impostato tramite la funzione \textit{process\_flag}, che verrà anch'essa spiegata in dettaglio nella sezione successiva.
\subfile{./LavoroLuca/EstensioneCauder/estensione}
\end{document}
