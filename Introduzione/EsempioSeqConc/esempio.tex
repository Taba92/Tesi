\documentclass[introduzione.tex]{subfiles}

\begin{document}
\begin{figure}[H]
\begin{multicols}{2}
a)Codice sequenziale\lstinputlisting{./Introduzione/EsempioSeqConc/exampleSeq.erl}
Il flusso del codice è sempre univocamente definito ed uguale ad ogni riesecuzione per l'input 3.
\vfill\null
\columnbreak
b)Codice concorrente\lstinputlisting{./Introduzione/EsempioSeqConc/exampleCunc.erl}
Il flusso del codice è univocamente definito solo fino al \textit{case} mentre, dopo la \textit{spawn}, può essere eseguita prima la stampa di "A" poi di "B", \textit{o viceversa} in base ad un certo \textit{scheduling}, \textit{cambiando potenzialmente ad ogni riesecuzione}.
\end{multicols}
\caption{Esempio codice sequenziale e concorrente}
\label{seqconc}
\end{figure}
\end{document}
