\documentclass[Tesi.tex]{subfiles}

\begin{document}

\section{BACKGROUND}
Come già anticipato nell'introduzione, il calcolo reversibile permette di eseguire computazioni sia in avanti che all'indietro, permettendo il recupero degli stati precedenti.
Esso trova origini in fisica, più precisamente nel \textit{Principio di Landauer} \cite{landauer}, affermando che l'eliminazione di un bit di informazione produce una quantità di calore che non può essere diminuita oltre un determinato limite.
Il principio di Landauer descrive il \textit{limite di Landauer} W, che fissa il minimo ammontare di energia che serve per cambiare un bit di informazione ed è formulato come segue: 
\[
W=kT ln 2
\]
dove
\begin{itemize}
	\item k è la costante di Boltzmann (1,380649x10-23 J $K^{-1}$)
	\item T è la temperatura assoluta del circuito in kelvin
	\item ln 2 è il logaritmo naturale di 2
\end{itemize}
Nella realtà quotidiana la quantità di energia rilasciata da una macchina per un qualsiasi processo di elaborazione dati è decisamente maggiore di molti ordini di grandezza rispetto al limite di Landauer.
La reversibilità fisica permetterebbe quindi di ottenere risultati senza rilascio di calore, ma come affermato sempre da Landauer \cite{wikipedia}, \textit{per poter essere reversibile fisicamente un calcolatore deve essere anche logicamente reversibile}.
Da allora il calcolo reversibile ha suscitato interesse in svariati campi, quali il design di hardware convenzionale e quantistico, biologia computazionale, sviluppo software ecc..\cite{book}.
Questa tesi si focalizzerà nell'applicazione del calcolo reversibile al debugging.
Articolerò il background in 2 step:
\begin{enumerate}
	\item Quali tecniche servono per applicare la reversibilità al debugging
	\item Implementare queste tecniche su uno strumento software reale
\end{enumerate}
\subfile{./Background/ReversibilitaToDebugging/reversibilita}
\subfile{./Background/Cauder/cauder}
\end{document}
