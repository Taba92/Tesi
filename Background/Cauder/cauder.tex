\documentclass[background.tex]{subfiles}

\begin{document}
\subsection{Cauder}
In questa sottosezione introduco Cauder ovvero: 
	\begin{enumerate}
		\item La sintassi che supporta, ponendo enfasi sulle funzioni rilevanti per questa tesi.
		\item Le strutture dati che manipola.
		\item La semantica indotta dalla sua sintassi e come modifica le strutture dati.
	\end{enumerate}
Per tutta questa sottosezione faccio riferimento a \cite{lanese19}.\\
\textbf{1) La sintassi supportata:}\\
La sintassi del linguaggio può essere trovata nella Figura 2 \ref{fig2}. Un modulo è una sequenza di definizioni di funzioni, dove ogni nome di funzione f / n (atom / arity) ha una definizione associata della forma $\displaystyle fun (X_{1},..., X_{n}) \xrightarrow{body} e$.
Il corpo di una funzione è \underline{un'espressione}, che può includere variabili, letterali, nomi di funzioni, liste, tuple, chiamate a funzioni predefinite, principalmente operatori aritmetici e relazionali, applicazioni di funzioni, espressioni case, associazioni let e espressioni di ricezione; inoltre, consideriamo anche le funzioni \textit{spawn}, \textit{"!"} (per inviare un messaggio) e \textit{self()} (ritorna il pid del processo chiamante) che in realtà sono incorporati nel linguaggio Erlang.
\begin{figure}[H]
  $\displaystyle
  \begin{array}{rcl@{~~~~~~}l}
    \mathit{module} & ::= & \mathsf{module} ~ Atom = %%[fname_1,\ldots,fname_n] =
    \mathit{fun}_1,\ldots,\mathit{fun}_n\\
    {\mathit{fun}} & ::= & \mathit{fname} = \mathsf{fun}~(X_1,\ldots,X_n) \to expr \\
    {\mathit{fname}} & ::= & Atom/Integer \\
    lit & ::= & Atom \mid Integer \mid Float \mid \nil \\
    expr & ::= & \mathit{Var} \mid lit \mid \mathit{fname} \mid [expr_1|expr_2]
                 \mid   \{expr_1,\ldots,expr_n\} \\
    & \mid & \mathsf{call}~expr~(expr_1,\ldots,expr_n) 
    \mid \mathsf{apply}~expr~(expr_1,\ldots,expr_n) \\
    & \mid &
    \mathsf{case}~expr~\mathsf{of}~clause_1;\ldots;clause_m~\mathsf{end}\\
    & \mid & \mathsf{let}~\mathit{Var}=expr_1~\mathsf{in}~expr_2 
    \mid \mathsf{receive}~clause_1;\ldots;clause_n~\mathsf{end}\\
    & \mid & \mathsf{spawn}(expr,[expr_1,\ldots,expr_n])  
     \mid expr_1 \:!\: expr_2 \mid \mathsf{self}()\\
    clause & ::= & pat ~\mathsf{when}~expr_1 \to expr_2
    \\
    pat & ::= & \mathit{Var} \mid lit \mid [pat_1|pat_2] \mid
    \{pat_1,\ldots,pat_n\} \\
  \end{array}
  $
\caption{Regole sintattiche del linguaggio} 
\label{fig2}
\end{figure}
Di seguito spiego con maggior dettaglio solo alcune \textit{expressioni}.
Il motivo di ciò risiede nel fatto che l'estensione di Cauder risulta mostrare analogie, più o meno rilevanti, con queste regole sintattiche e le loro regole semantiche.\\
\begin{itemize}
	\item $\displaystyle \mathsf{spawn}(expr,[expr_{1},...,expr_{n}])\xrightarrow{return}SpawnedPid$: La chiamata alla funzione $\mathsf{spawn}$ crea un nuovo processo, che inizia con la valutazione di $\displaystyle \mathsf{apply}(expr,[expr_{1},...,expr_{n}])$. Ritorna il \textit{pid} del processo appena generato.
	\item $\displaystyle expr_{1}~!~expr_{2}\xrightarrow{return}expr_{2}$: La chiamata alla funzione send (!) invia il messaggio \textit{expr2} al processo con \textit{pid $expr_{1}$}. Ritorna il messaggio inviato ovvero \textit{$expr_{2}$}.
	\item $\displaystyle \mathsf{receive}~clause_{1};...;clause_{n}\xrightarrow{return}expr_{2}$ di clause matching: L'espressione $\mathsf{receive}$ attraversa i messaggi nella coda dei messaggi del processo finché uno di essi non corrisponde a un ramo nell'istruzione di ricezione; dovrebbe trovare il primo messaggio v nella coda tale che $\exists$ clause $\in$ \{$clause_{1}$;...;$clause_{n}$\} tale che v matcha \textit{pat} di clause $\wedge$ \textit{$expr_{1}$} di clause==$\mathsf{true}$; ritorna \textit{$expr_{2}$} di clause, con l'ulteriore effetto collaterale di eliminare il messaggio v dalla coda dei messaggi del processo. Se non sono presenti messaggi corrispondenti nella coda, il processo sospende la sua esecuzione finché non arriva un messaggio corrispondente.
\end{itemize}
Nel punto relativo alla semantica ometto la spiegazione della receive, in quanto è interessante solo dal punto di vista comportamentale.\\
\textbf{2) Le strutture dati di Cauder:}\\
In questa sezione vado a definire le strutture dati gestite da Cauder.
Nella figura sottostante mostro come lo stato di Cauder viene rappresentato graficamente.
Ad ogni struttura dati che introduco, associerò come essa viene rappresentata internamente in Cauder e come venga rappresentata graficamente, facendo riferimento alla figura sottostante.
\begin{figure}[H]
	\centerline{\includegraphics[scale=0.5]{./Background/Cauder/Imgs/CauderStato}}
	\caption{Lo stato di Cauder.}
	\label{fig3}
\end{figure}
Un \textit{sistema} in esecuzione viene denotato tramite $\Gamma$;$\Pi$, dove $\Gamma$ denota la mailbox global ovvero \underline{l'insieme} dei \textit{messaggi inviati} in attesa di essere consegnati, mentre $\Pi$ denota \underline{l'insieme} dei \textit{processi} nel sistema.
Il \textit{sistema} viene implementato in Cauder tramite un \textit{record sys} contenente i seguenti campi:
	\begin{itemize}
		\item \textit{\textbf{sched}}: denota la tipologia di scheduling delle azioni da utilizzare. Cauder utilizza due politiche di scheduling:
			\begin{itemize}
				\item RANDOM: L'azione viene scelta casualmente \textit{tra tutte le azioni possibili} del sistema.
				\item PRIO\_RANDOM: L'azione viene scelta casualmente \textit{tra tutte le azioni effettuabili dai processi} e se non ce ne sono viene scelta casualmente \textit{tra tutte le azioni effettuabili per la schedulazione dei messaggi}.
			\end{itemize}
			Nel punto della semantica verrà spiegato il perchè è necessaria una schedulazione dei messaggi.
		\item \textit{\textbf{msgs}}: denota la mailbox globale ovvero $\Gamma$. Nella figura \ref{fig3} viene rappresentata da \textit{GM}.
		\item \textit{\textbf{procs}}: denota l'insieme dei processi ovvero $\Pi$. Nella figura \ref{fig3} viene rappresentata dalla lista \textit{======Proc.N:fun/M====}.
		\item \textit{\textbf{trace}}: denota le azioni effettuate nel sistema. Nella figura \ref{fig3} vengono visualizzate nel \textit{tab Trace}.
		\item \textit{\textbf{roll}}: denota le operazioni di \textit{roll} effettuate nel sistema. Nella figura \ref{fig3} vengono visualizzare nel \textit{tab Roll log}.
	\end{itemize}
Illustrato la struttura di un \textit{sistema} mi soffermo sui campi \textit{msgs} e \textit{procs}, in quanto, \textbf{per motivi diversi e che verrano spiegato nella sezione relativa all'estensione}, risultano essere rilevanti per questa tesi.
Il campo \textit{sched} verrà menzionato nella parte dell'estensione, ma non approfondito in quanto non troppo rilevante per questa tesi.
\textit{Msgs} risulta essere implementata tramite una \textit{lista} di \textit{msg}.\\
Formalmente \textit{msg} è una tripla del tipo \textit{(pid\_dest,value,time)}. Il campo \textit{time} risulta essere necessario in quanto discrimina quel messagio specifico, dato che potre avere più messaggi con lo stesso valore e destinatario.
Banalmente, un \textit{msg} viene implementato tramite un \textit{record msg} contenente i campi sopra citati.
Analogalmente a \textit{msgs}, \textit{procs} viene implementato tramite una \textit{lista} di \textit{proc}.
Formalmente \textit{proc} viene denotato tramite \textit{$\langle$ p,$\theta$,e,h,lm$\rangle$}, dove:
	\begin{itemize}
		\item \textit{p}: rappresenta il pid del processo.
		\item \textit{$\theta$}: rappresenta l'ambiente del processo, ovvero \textit{l'insieme} dei bindings.
		\item \textit{e}: rappresenta l'espressione da valutare.
		\item \textit{h}: rappresenta \textit{l'history} del processo, ovvero la \textit{sequenza} di tutte le azioni che il processo ha effettuato.
		\item \textit{lm}: rappresenta la mailbox del processo, ovvero la \textit{sequenza} di tutti i messaggi \textit{schedulati} il cui destinatario è il processo.
	\end{itemize}
\textit{Proc} viene implementata tramite un \textit{record proc} con i seguenti campi e farò riferimento alla figura \ref{fig3} per le visualizzazioni:
	\begin{itemize}
		\item \textit{pid}: implementa \textit{p} ed è visualizzato tramite \textit{N} in \textit{======Proc.N:fun/M====}.
		\item \textit{hist}: implementa \textit{h} ed è visualizzato tramite \textit{H}.
		\item \textit{env}: implementa \textit{$\theta$} ed è visualizzato tramite \textit{ENV}.
		\item \textit{exp}: implementa \textit{e} ed è visualizzato tramite \textit{EXP}.
		\item \textit{mail}: implementa \textit{lm} ed è visualizzato tramite \textit{LM}.
	\end{itemize}
	Il \textit{record proc} possiede campi aggiuntivi ma che non menziono dato la non rilevanza per questa tesi.\\
\textbf{2) La semantica di Cauder:}\\
In questo punto introduco la semantica derivante dalla sintassi di Cauder. Più precisamente mostro solo un sottoinsieme di queste regole, in quanto le altre non sono rilevanti per questa tesi.
Evidenzio \textbf{4} regole, che vado ad illustrare, lavorando su un processo generico P \textit{che esegue la regola descritta}, ovvero $\displaystyle \langle p,\theta,e,h,lm\}$ e su un sistema generico S, ovvero $\Gamma$;$\Pi$.\\
Essendo un linguaggio funzionale ad ogni passo creo delle nuove strutture dati sulla base delle vecchie, costruendo nuovi stati a partire dai precedenti.\\
Come già accennato in precedenza, sia gli insiemi che le sequenze vengono implementati come \textit{liste}. A livello implementativo, per l'aggiunta di un elemento in una lista, uso la notazione [H$\mid$L], ovvero creo una nuova lista aggiungendo \textit{H} in testa alla lista \textit{L}.
Nella spiegazione della semantica, per la modifica degli insiemi utizzo la notazione insiemistica mentre per le sequenze utilizzo direttamente la notazione implementativa.
Ciò viene fatto per tenere evidenziato cosa è un insieme e cosa è una sequenza.
Denoto le costanti tramite $\mathsf{valore\_atomo}$, mentre per riferirmi ai campi delle strutture dati, utilizzo la notazione punto (esempio: flag di un processo P = P.f).\\
\textbf{Se non esplicitamente utilizzata, allora mi riferisco ai campi di un processo}.
Nel punto 1 ho posto enfasi solo su 3 regole sintattiche, ovvero la $\mathsf{spawn}$ per generare processi, la $\mathsf{send}$ per inviare un messaggio e la $\mathsf{receive}$ per elaborare un messaggio presente nella local mailbox del processo.
\textbf{Ma non è definito come un messaggio venga recapitato dalla global mailbox alla local mailbox, dato l'asincronicità illustrata all'inizio di questa sezione}. La quarta regola semantica (\textit{rule sched}) serve per simulare questa asincronicità.\\
\textbf{SEMANTICA IN AVANTI}:
\begin{itemize}
	\item \textit{rule spawn}: Informalmente questa regola va ad inserire in $\Pi$ un \textit{proc} vuoto. Formalmente:\\
	Sia P'=$\displaystyle \langle p',\theta'=\empty,e'=\mathsf{apply}(fun(expr_{1}...expr_{n})),h'=[],lm=[]\rangle$ il processo spawnato.\\
	Sia h''=$\displaystyle [\{\mathsf{spawn},\theta,e,p'\} \mid h]$.\\
	Sia P''=$\displaystyle \langle p,\theta'',e'',h'',lm\rangle$.\\
	Il sistema risultante da questo passaggio sarà S'=$\displaystyle \Gamma;\Pi\setminus\{P\}\cup\{P',P''\}$.
	\item \textit{\textbf{rule send}}: Informalmente questa regola va ad inserire in $\Gamma$ un \textit{msg}. Formalmente:\\
	Sia msg=\textit{(dest\_pid,payload,time)}.\\
	Sia h'=$\displaystyle [\{\mathsf{send},\theta,e,msg\} \mid h]$.\\
	Sia P'=$\displaystyle \langle p,\theta',e'',h',lm \rangle$.
	Il sistema risultante da questo passaggio sarà S'=$\displaystyle \Gamma\cup\{msg\};\Pi\setminus\{P\}\cup\{P'\}$.
	\item \textit{\textbf{rule sched}}: 
	Sia msg=\textit{(dest\_pid,payload,time)} il messaggio selezionato per la schedulazione.
	Sia lm'=[msg $\mid$ lm].
	Sia P'=$\displaystyle \langle p,\theta,e,h,lm'\rangle$.
	Il sistema risultante da questo passaggio sarà S'=$\displaystyle \Gamma\setminus\{msg\};\Pi\setminus\{P\}\cup\{P'\}$.
\end{itemize}
\textbf{SEMANTICA ALL'INDIETRO}:
	\begin{itemize}
		\item \textit{\textbf{rule spawn}}: Informalmente questa regole va ad eliminare da $\Pi$ un \textit{proc} spawnato. Come precondizione per effettuare l'undo della $\mathsf{spawn}$, bisogna che il processo spawnato sia vuoto. Più formalmente bisogna che P'=$\displaystyle \langle p',\theta',e',h'=[],lm=[]\rangle$. Se non fosse rispettata questa precondizione si rischierebbe di violare il Loop Lemma.
		Ipotizziamo di avere 3 processi cosi definiti, in cui il processo 1 spawna il secondo e il secondo spawna il terzo.
		Abbiamo una situazione del genere:
		\begin{enumerate}
			\item $\displaystyle \langle p',\theta',e',h',lm' \rangle$ con h'=$\displaystyle [\{\mathsf{spawn},\theta,e,p''\} \mid h]$.
			\item  $\displaystyle \langle p'',\theta'',e'',h'',lm''\rangle$ con h''=$\displaystyle[\{\mathsf{spawn},\theta,e,p'''\} \mid h]$.
			\item il terzo processo non è rilevante.
		\end{enumerate}
		Ora facciamo l'undo della $\mathsf{spawn}$ del secondo processo per poi rifarla.
		Ci ritroviamo nella seguente situazione:
			\begin{enumerate}
			\item $\displaystyle \langle p',\theta',e',h',lm' \rangle$ con h'=$\displaystyle [\{\mathsf{spawn},\theta,e,p''\} \mid h]$.
			\item $\displaystyle \langle p'',\theta'',e'',h'',lm'' \rangle$ con h''=[] ed lm''=[].
			\item il terzo processo non è rilevante.
		\end{enumerate}
		Abbiamo fatto l'undo di un'azione per poi rifarla e siamo finiti su uno stato diverso, violando il Loop Lemma.\\
		Posta vi sia questa condizione, l'undo di una spawn si formalizza così:
		Sia h=$\displaystyle [\{\mathsf{spawn},\theta,e,p'\} \mid h']$.\\
		Sia P''=$\displaystyle \langle p,\theta'' = \theta,e''=e,h',lm \rangle$.\\
		Allora il sistema evolve in S'=$\displaystyle \Gamma;\Pi\setminus\{P',P\}\cup\{P''\}$.
		\item \textit{\textbf{rule send}}: Informalmente questa regola elimina un \textit{msg} da $\Gamma$. Come precondizione per effettuare l'undo di una $\mathsf{send}$, bisogna che il messaggio \textit{msg} sia presente in $\Gamma$. Se così non fosse si rischierebbe di violare il Loop Lemma.
		Ipotizziamo questo scenario:
		Un messaggio \textit{msg=(p',val,time)} è presente nella local mailbox di p' ( è già stato schedulato). Quindi ho questi due processi:
			\begin{itemize}
				\item P'=$\displaystyle \langle p',\theta',e',h',lm'\rangle$ con lm'=[msg  $\mid$ lm] che è il destinatario del messaggio.
				\item $\displaystyle \Gamma'=\Gamma\setminus\{msg\}$.
			\end{itemize}
		Se facessi l'undo della send per poi rifarla subito dopo mi ritroverei in questa situazione:
			\begin{itemize}
				\item P'=$\displaystyle \langle p',\theta',e',h',lm'\rangle$ con lm'=[msg  $\mid$ lm] che è il destinatario del messaggio.
				\item $\displaystyle \Gamma'=\Gamma\cup\{msg\}$.
			\end{itemize}
		In sostanza mi troverei con un messaggio duplicato, che chiaramente viola il Loop Lemma.\\
		Posto vi sia questa condizione, l'undo di una send si formalizza in questo modo:
		Sia \textit{msg=(p',val,time)}.
		Sia h'=$\displaystyle [\{\mathsf{send},\theta,e,msg\} \mid h]$.
		Sia P'=$\displaystyle \langle p,\theta'=\theta,e'=e,h,lm \rangle$.
		Allora il sistema evolve in S'=$\displaystyle \Gamma\setminus\{msg\};\Pi\setminus\{P\}\cup\{P'\}$.
		\item \textit{\textbf{rule sched}}: Informalmente questa regola elimina un \textit{msg} dalla local mailbox del processo destinatario, per reinserirla in $\Gamma$.
		Come precondizione per effettuare l'undo di una sched, bisogna che il \textit{msg} sia in \underline{testa} alla local mailbox del processo ricevente. Se così non fosse si rischierebbe di violare il Loop Lemma.
		Ipotizziamo di voler invertire la \textit{sched} di \textit{msg} e di essere in questa situazione:\\
		Sia \textit{lm'=[msg',msg $\mid$ lm]} la local mailbox del processo destinatario di \textit{msg}.\\
		Se ora facessi l'undo della sched di \textit{msg} per poi rifare subito dopo la sched di \textit{msg} mi ritroverei in quest'altra situazione:\\
		\textit{lm'=[msg,msg' $\mid$ lm]}.
		Ho fatto l'undo di un'azione per poi rifarla e mi sono ritrovato in uno stato diverso, violando chiaramente il Loop Lemma.
		Posto vi sia questa condizione, l'undo della \textit{sched} si formalizza in questo modo:\\
		Sia \textit{lm=[msg $\mid$ lm']} la local mailbox del processo destinatario di \textit{msg}.\\
		Sia P'=$\displaystyle \langle p,\theta,e,h,lm' \rangle$.
		Allora il sistema evolve in S'=$\displaystyle \Gamma\cup\{msg\};\Pi\setminus\{P\}\cup\{P'\}$.
	\end{itemize}
\textbf{ROLL}:\\
	Informalmente, l'operatore \textit{roll} di un'azione si va a ridurre ad un passo della semantica all'indietro della relativa azione, \textbf{andando a creare in primis le condizioni per cui poter effettuare quel passo} (annullando tutte le sue conseguenze), per poi effettuare quel passo.
	Indico con \textit{back\_step(S,p)} una funzione che dato uno stato \textit{S} e un pid \textit{p}, osserva l'elemento in cima alla history \textit{h} del processo con pid \textit{p} ed effettua un passo all'indietro su di esso, ritornando il nuovo stato S'.
	Indico invece con \textit{back\_rule\_xxx} l'undo della \textit{rule\_xxx}.
	\begin{itemize}
		\item \textit{\textbf{spawn}}: A livello informale la \textit{roll spawn} va a "svuotare" un processo (annullando tutte le sue conseguenze ricorsivamente), per poi infine fare l'undo della spawn di esso. Algoritmicamente:
		Sia $\displaystyle h=[\{\mathsf{spawn},\theta,e,p'\} \mid h']$.\\
		Sia P' il processo con pid \textit{p'}.
		\begin{algorithm}[H]
		\caption{roll\_spawn(State,p,p')}
		\begin{algorithmic}
		\IF {P'.lm==[] $\wedge$ P'.h==[]} 
			\RETURN back\_rule\_spawn(State,p)
		\ELSE
			\STATE State'=back\_step(State,p')
			\STATE roll\_spawn(State',p,p')
		\ENDIF
		\end{algorithmic}
		\end{algorithm}
		\item \textit{\textbf{rule send}}: 
		A livello informale la \textit{roll send} va a riportare un \textit{msg} all'interno di $\Gamma$, per poi effettuare l'undo della send. Diversamente dalla \textit{roll spawn}, la \textit{roll send} deve anche valutare le conseguenze derivate dalla schedulazione del messaggio oltre che le conseguenze del processo destinatario. Algoritmicamente:
		Sia \textit{msg=(p',val,time)}.
		Sia h=$\displaystyle [\{\mathsf{send},\theta,e,msg\} \mid h_{old}]$.
		Sia \textit{lm'} la localmail box di P'.
		Sia 
		\begin{algorithm}[H]
		\caption{roll\_send(State,p,p',time)}
		\begin{algorithmic}
		\IF { msg $\in$ $\Gamma$} 
			\RETURN back\_step(State,p)
		\ELSE
			\IF { lm'==[msg $\mid$ $lm_{rest}$]} 
				\STATE State'=back\_rule\_sched(State,msg)
				\RETURN roll\_send(State',p,p',time)
			\ELSE
				\STATE State'=back\_step(State,p')
				\RETURN roll\_send(State',p,p',time)
			\ENDIF
		\ENDIF
		\end{algorithmic}
		\end{algorithm}
	\end{itemize}
\end{document}